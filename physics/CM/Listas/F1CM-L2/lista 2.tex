% arara: xelatex: {synctex: true}
% arara: indent: {overwrite: yes}
\documentclass[]{IMTexam}

\usepackage[enums]{IMTtikz}

\givecredits
\author{Isabella B.}
\USPN{11810773}
\date{}
\lecture{Física I} % disciplina
\lcode{CM0112}
\hwtype{Resolução} % o que é
\examname{Lista 2} % prova

\begin{document}

\maketitle

\begin{questions}

	\question Uma corda com densidade de massa uniforme, tem massa $ M $ e está amarrada em 2 árvores. Os extremos da corda estão na mesma altura e cada extremo faz um ângulo $\theta$ como mostrado na figura.
	\begin{parts}
		\part Ache a tensão nos extremos.

		\part Ache a tensão no meio da corda.
	\end{parts}

	\question Uma superfície sem atrito tem forma de uma função $ y(x) $ genérica na qual os pontos extremos tem a mesma altura. Uma corda com massa uniforme está apoiada na superfície, prove que a corda está em equilíbrio para qualquer que seja a função $ y(x) $.

	\question Uma corda de comprimento $ l $ e densidade $\rho$ está suspensa verticalmente.

	\begin{parts}
		\part Ache a tensão ao longo da corda.

		\part A mesma corda, agora repousa sobre um plano inclinado de ângulo $\theta$, o topo da corda está parafusado ao plano. Sabendo que o coeficiente de atrito é $\mu$, ache a tensão no topo da corda.

	\end{parts}

	\question Encontre a tensão em cada uma das 3 cordas ideais da figura, sabendo que o objeto $ W $ possui peso $ P = mg $.

	\begin{solution}

	\end{solution}

	\question Uma partícula de massa $ m $ segue um caminho no eixo $ xy $ dado pelas equações \[ x = A(\alpha t- \sin \alpha t)\quad\text{e}\quad y = A(1 - \cos \alpha t) \]
	Faça o gráfico do caminho e encontre o vetor força dependente do tempo. Você conhece algum exemplo que produz esse movimento?

	\begin{solution}

	\end{solution}

	\question Na figura o bloco $ C $ está caindo. Encontre a aceleração de cada uma das massas sabendo que tanto as polias quanto as cordas são ideais.

	\begin{solution}

	\end{solution}

	\question Uma partícula move-se num plano, sua posição pode ser descrita por coordenadas retangulares (x, y) ou por coordenadas polares (r, $\theta$), onde $x = r \cos \theta$ e $y = r \sin \theta$

	\begin{parts}
		\part Calcule $ a_x $ e $ a_y $ como derivadas temporais de $r \cos \theta$ e $r \sin \theta$.

		\begin{solution}

		\end{solution}

		\part Mostre que as componentes da aceleração em coordenadas polares são:
		\begin{gather*}
			a_r = a_x \cos \theta + a_y \sin \theta\\
			a_\theta = -a_x \sin \theta + a_y \cos \theta
		\end{gather*}

		\begin{solution}

		\end{solution}

		\part Mostre que o vetor aceleração é dado por: \[ \vec{a} = \del{\ddot{r} - r\,\dot{\theta}^{2}}\rhat + \del{2\dot{r}\,\dot{\theta} + r\,\ddot{\theta}}\that \]

		\begin{solution}

		\end{solution}

	\end{parts}

	\question Uma corda está envolta de um cilindro fixo como mostrado na figura. O coeficiente de atrito entre a corda e o cilindro é $\mu$. O ângulo $ \theta_0 = \pi/3 $ define o arco do cilindro envolto pela corda inicialmente. Um gato está puxando uma extremidade da corda com força $ F $ enquanto $ 10 $ ratos equilibram a corda no limiar do deslizamento $ F $ aplicando cada um uma força de $ F/100 $.

	\begin{parts}
		\part A mínima força pra equilibrar a corda depende do diâmetro do cilindro? Por que?

		\begin{solution}

		\end{solution}

		\part Qual é o ângulo mínimo $\theta$ para que 1 rato impeça o gato de vencer o cabo de guerra?

		\begin{solution}

		\end{solution}
	\end{parts}

	\question A massa m está conectada a um eixo vertical que gira por meio de duas cordas de comprimento $ l $, onde cada corda faz um ângulo de \ang{45} como é mostrado. O eixo e a massa estão girando com velocidade angular $ \omega $. Qual é a tensão em cada uma das cordas?

	\begin{solution}

	\end{solution}

	\question Uma corda com densidade de massa constante tem massa $ M $, comprimento $ L $ e está girando com velocidade angular uniforme $ \omega $. Qual e a tensão no meio da corda? Não considere a gravidade.

	\begin{solution}

	\end{solution}

	\question As massas deslizam sobre um plano inclinado, há atrito entre as massas e o plano, $\mu_1$ e $\mu_2$ respectivamente. Determine a aceleração das duas massas e o valor força $ F $ que faz $ m_2 $ sobre $ m_1 $. Discuta os movimentos possíveis do sistema e as condições necessárias para que a força $ F $ exista.

	\begin{solution}

	\end{solution}

	\question (DESAFIO) Uma bola de vôlei é jogada para cima com velocidade $ v_0 $. Assuma que a força de arrasto é $ F = -\alpha\,v $.

	\begin{parts}
		\part Qual é a velocidade final $ v_f $ da bola quando ela atinge o chão?

		\begin{solution}

		\end{solution}

		\part A bola passa mais ou menos tempo no ar se comparado com um lançamento sem resistência do ar?

		\begin{solution}

		\end{solution}
	\end{parts}

	\question (DESAFIO) Um dico de massa $ M $ e raio $ R $ está pendurada sobre uma corda sem massa como mostra a figura:
	\begin{parts}
		\part Assumindo que não há atrito, qual é a tensão na corda?

		\begin{solution}

		\end{solution}
		\part Assumindo que há atrito de coeficiente $\mu$, qual é a menor tensão possível no ponto mais baixo da corda?

		\begin{solution}

		\end{solution}
	\end{parts}

\end{questions}
\end{document}
